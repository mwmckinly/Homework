
\documentclass{article}
\usepackage[legalpaper, margin=0.5in]{geometry}

\usepackage{textcomp}
\usepackage{fontspec}
\usepackage{unicode-math}
\everymath{\displaystyle\scriptstyle}
\usepackage{cprotect}
\setmainfont{Latin Modern Roman}
\setmathfont{Latin Modern Math}

\usepackage{hyperref}

\usepackage{amsmath}

\relpenalty=10000
\binoppenalty=10000

\setlength{\parindent}{0pt}

\begin{document}

\newlength{\partheight}
\setlength{\partheight}{0.2375\textheight}

\begin{minipage}[t]{\textwidth}
\centering
{\Large \textbf{Homework 1.5-2.3}}
\end{minipage}

\begin{minipage}[t][\partheight]{\textwidth}
Problems 33 through 37 illustrate the application of linear first-order
differential equations to mixture problems.
\\\\
\textbf{1.5.33.}\quad A tank contains 1000 liters (L) of a solution consisting of 100 kg of
salt dissolved in water. Pure water is pumped into the tank at the rate
of 5 L/s, and the mixture---kept uniform by stirring---is pumped out at
the same rate. How long will it be until only 10 kg of salt remains in
the tank?

\end{minipage}
\par
\begin{minipage}[t][\partheight]{\textwidth}
\textbf{1.5.37.}\quad A 400-gal tank initially contains 100 gal of brine containing 50 lb of
salt. Brine containing 1 lb of salt per gallon enters the tank at the
rate of 5 gal/s, and the well-mixed brine in the tank flows out at the
rate of 3 gal/s. How much salt will the tank contain when it is full of
brine?

\end{minipage}
\par
\begin{minipage}[t][\partheight]{\textwidth}
Separate variables and use partial fractions to solve the initial value
problems in Problems 1--8. Use either the exact solution or a
computer-generated slope field to sketch the graphs of several solutions
of the given differential equation, and highlight the indicated
particular solution.
\\\\
\textbf{2.1.3.}\quad {$\displaystyle \frac{dx}{dt} = 1 - x^{2},x(0) = 3$}

\end{minipage}
\par
\begin{minipage}[t][\partheight]{\textwidth}
\textbf{2.1.5.}\quad {$\displaystyle \frac{dx}{dt} = 3x(5 - x),x(0) = 8$}

\end{minipage}
\par
\begin{minipage}[t][\partheight]{\textwidth}
\textbf{2.1.12.}\quad {Population growth} The time rate of change of an alligator population
\texttt{P} in a swamp is proportional to the square of \texttt{P}. The
swamp contained a dozen alligators in 1988, two dozen in 1998. When will
there be four dozen alligators in the swamp? What happens thereafter?

\end{minipage}
\par
\begin{minipage}[t][\partheight]{\textwidth}
\textbf{2.1.13.}\quad {Birth rate exceeds death rate} Consider a prolific breed of rabbits
whose birth and death rates, {$\displaystyle \beta$} and {$\displaystyle \delta\text{,}$} are
each proportional to the rabbit population {$\displaystyle P = P(t)\text{,}$} with
{$\displaystyle \beta > \delta\text{.}$}

\hspace*{1.5em}\textbf{(a)} Show that $\displaystyle P(t) = \frac{P_{0}}{1 - kP_{0}t}, k\text{constant}\text{.}$ Note that {$\displaystyle \left. P(t)\rightarrow + \infty \right.$} as
{$\displaystyle \left. t\rightarrow 1/\left( kP_{0} \right)\text{.} \right.$} This
is doomsday.

\hspace*{1.5em}\textbf{(b)} Suppose that {$\displaystyle P_{0} = 6$} and that there
are nine rabbits after ten months. When does doomsday occur?

\end{minipage}
\par
\begin{minipage}[t][\partheight]{\textwidth}
\textbf{2.1.26.}\quad {Constant death rate} A population \texttt{P}(\texttt{t}) of small
rodents has birth rate {$\displaystyle \beta = (0.001)P$} (births per month per
rodent) and \emph{constant} death rate {$\displaystyle \delta\text{.}$} If
{$\displaystyle P(0) = 100$} and {$\displaystyle P\text{'}(0) = 8\text{,}$} how long (in months)
will it take this population to double to 200 rodents?
(\emph{Suggestion:} First find the value of {$\displaystyle \delta$}.)

\end{minipage}
\par
\begin{minipage}[t][\partheight]{\textwidth}
\textbf{2.1.27.}\quad {Constant death rate} Consider an animal population
\texttt{P}(\texttt{t}) with constant death rate {$\displaystyle \delta = 0.01$}
(deaths per animal per month) and with birth rate {$\displaystyle \beta$}
proportional to \texttt{P}. Suppose that {$\displaystyle P(0) = 200$} and
{$\displaystyle P\text{'}(0) = 2\text{.}$}

\hspace*{1.5em}\textbf{(a)} When is
{$\displaystyle P = 1000\text{?}$}

\hspace*{1.5em}\textbf{(b)} When does doomsday occur?

\end{minipage}
\par
\begin{minipage}[t][\partheight]{\textwidth}
In Problems 1 through 12 first solve the equation {\(f(x) = 0\)} to find
the critical points of the given autonomous differential equation
{\(dx/dt = f(x)\text{.}\)} Then analyze the sign of
\texttt{f}(\texttt{x}) to determine whether each critical point is
stable or unstable, and construct the corresponding phase diagram for
the differential equation. Next, solve the differential equation
explicitly for \texttt{x}(\texttt{t}) in terms of \texttt{t}. Finally,
use either the exact solution or a computer-generated slope field to
sketch typical solution curves for the given differential equation, and
verify visually the stability of each critical point.
\\\\
\textbf{2.2.1.}\quad {$\displaystyle \frac{dx}{dt} = x - 4$}

\end{minipage}
\par
\begin{minipage}[t][\partheight]{\textwidth}
\textbf{2.2.3.}\quad {}{$\displaystyle \frac{dx}{dt} = x^{2} - 4x$}

\end{minipage}
\par
\begin{minipage}[t][\partheight]{\textwidth}
\textbf{2.2.7.}\quad {$\displaystyle \frac{dx}{dt} = (x - 2)^{2}$}

\end{minipage}
\par
\begin{minipage}[t][\partheight]{\textwidth}
\textbf{2.2.26.}\quad If {$\displaystyle 4h = kM^{2}\text{,}$} show that typical solution curves look as
illustrated in . Thus if {$\displaystyle x_{0} \geqq M/2\text{,}$} then
{$\displaystyle \left. x(t)\rightarrow M/2 \right.$} as
{$\displaystyle \left. t\rightarrow + \infty\text{.} \right.$} But if
{$\displaystyle x_{0} < M/2\text{,}$} then {$\displaystyle x(t) = 0$} after a finite period of
time, so the lake is fished out. The critical point {$\displaystyle x = M/2$} might
be called \emph{semistable}, because it looks stable from one side,
unstable from the other. 
\end{minipage}
\par
\begin{minipage}[t][\partheight]{\textwidth}
\textbf{2.2.27.}\quad If {$\displaystyle 4h > kM^{2}\text{,}$} show that {$\displaystyle x(t) = 0$} after a finite
period of time, so the lake is fished out (whatever the initial
population). {[}\emph{Suggestion:} Complete the square to rewrite the
differential equation in the form
{$\displaystyle dx/dt = - k\left\lbrack (x - a)^{2} + b^{2} \right\rbrack\text{.}$}
{}Then solve explicitly by separation of variables.{]} The results of
this and the previous problem (together with ) show that
{$\displaystyle h = \frac{1}{4}kM^{2}$} is a critical harvesting rate for a logistic
population. At any lesser harvesting rate the population approaches a
limiting population \texttt{N} that is less than \texttt{M} (why?),
whereas at any greater harvesting rate the population reaches
extinction.

\end{minipage}
\par
\begin{minipage}[t][\partheight]{\textwidth}
\textbf{2.3.1.}\quad {No air resistance} Suppose that a crossbow bolt is shot straight upward
from the ground ({$\displaystyle y_{0} = 0$}) with initial velocity {$\displaystyle v_{0} = 49$}
(m/s). Then with {$\displaystyle g = 9.8$} gives $\displaystyle \frac{dv}{dt} = - 9.8,\text{so} v(t) = - (9.8)t + v_{0} = - (9.8)t + 49.$ Hence the bolt's height function \texttt{y}(\texttt{t}) is given by $\displaystyle y(t) = {\int{\left\lbrack - (9.8)t + 49 \right\rbrack dt = - (4.9)t^{2} + 49t + y_{0} = - (4.9)t^{2} + 49t\text{.}}}$ The bolt reaches its maximum height when
{$\displaystyle v = - (9.8)t + 49 = 0\text{,}$} hence when {$\displaystyle t = 5$} (s). Thus its
maximum height is $\displaystyle y_{\max} = y(5) = - (4.9)\left( 5^{2} \right) + (49)(5) = 122.5\left( \text{m} \right)\text{.}$ The bolt returns to the ground when
{$\displaystyle y = - (4.9)t(t - 10) = 0\text{,}$} and thus after 10 seconds aloft.

\end{minipage}
\par
\begin{minipage}[t][\partheight]{\textwidth}
\textbf{2.3.2.}\quad {Velocity-proportional resistance} We again consider a bolt shot
straight upward with initial velocity {$\displaystyle v_{0} = 49\text{m/s}$} from a
crossbow at ground level. But now we take air resistance into account,
with {$\displaystyle \rho = 0.04$} in . We ask how the resulting maximum height and
time aloft compare with the values found in .

\end{minipage}
\par
\begin{minipage}[t][\partheight]{\textwidth}
Problems 9 through 12 illustrate resistance proportional to the
velocity.
\\\\
\textbf{2.3.7.}\quad Suppose that a car starts from rest, its engine providing an
acceleration of {$\displaystyle 10\text{ft/s}^{2}\text{,}$} while air resistance
provides {$\displaystyle 0.1\text{ft/s}^{2}$} of deceleration for each foot per
second of the car's velocity.

\hspace*{1.5em}\textbf{(a)} Find the car's maximum
possible (limiting) velocity.

\hspace*{1.5em}\textbf{(b)} Find how long it takes the
car to attain 90\% of its limiting velocity, and how far it travels
while doing so.

\end{minipage}
\par
\begin{minipage}[t][\partheight]{\textwidth}
Problems 17 and 18 apply -- to the motion of a crossbow bolt.
\\\\
\textbf{2.3.9.}\quad A motorboat weighs 32,000 lb and its motor provides a thrust of 5000 lb.
Assume that the water resistance is 100 pounds for each foot per second
of the speed \texttt{v} of the boat. Then $\displaystyle 1000\frac{dv}{dt} = 5000 - 100v.$ If the boat starts from rest, what is the maximum velocity that it can
attain?

\end{minipage}
\par
\newpage
\begin{center}
{\Large \textbf{Answer Key}}
\end{center}
\vspace{1em}
\noindent\textbf{1.5.33.}
\par
\textbf{} After about 7 min 41 s

\vspace{1em}
\noindent\textbf{1.5.37.}
\par
\textbf{} 393:75 lb

\vspace{1em}
\noindent\textbf{2.1.3.}
\par
\textbf{} {\(x(t) = \frac{2 + e^{- 2t}}{2 - e^{- 2t}}\)}

\vspace{1em}
\noindent\textbf{2.1.5.}
\par
\textbf{} {\(x(t) = \frac{40}{8 - 3e^{- 15t}}\)}

\vspace{1em}
\noindent\textbf{2.1.12.}
\par
\textbf{} {\(P(t) = \frac{240}{20 - t}\)}

\vspace{1em}
\noindent\textbf{2.1.13.}
\par
\textbf{} {\(P(t) = \frac{180}{30 - t}\)}

\vspace{1em}
\noindent\textbf{2.1.26.}
\par
\textbf{} {\(50\ln\frac{9}{8} \approx 15.89\text{months}\)}

\vspace{1em}
\noindent\textbf{2.1.27.}
\par
\textbf{} (a) {\(100\ln\frac{9}{5} \approx 58.78\text{months}\)}\;;\;\;  (b)
{\(100\ln 2 \approx 69.31\text{months}\)}.

\vspace{1em}
\noindent\textbf{2.2.1.}
\par
\textbf{} Unstable critical point: {\(x = 4\;;\;\; \)} {\(x(t) = 4 + \left( x_{0} - 4 \right)e^{t}\)}

\vspace{1em}
\noindent\textbf{2.2.3.}
\par
\textbf{} Stable critical point: {\(x = 0\;;\;\; \)} unstable critical point:
{\(x = 4\;;\;\; \)} {\(x(t) = \frac{4x_{0}}{x_{0} + \left( {4 - x_{0}} \right)e^{4t}}\)}

\vspace{1em}
\noindent\textbf{2.2.7.}
\par
{}\textbf{} Semi-stable (see ) critical point: {\(x = 2\;;\;\; \)} {\(x(t) = \frac{\left( {2t - 1} \right)x_{0} - 4t}{tx_{0} - 2t - 1}\)}

\vspace{1em}
\noindent\textbf{2.3.2.}
\par
\hypertarget{solution}{%
\section{\texorpdfstring{{Solution}}{Solution}}} We substitute {\(y_{0} = 0,v_{0} = 49\text{,}\)} and
{\(v_{\tau} = - g/\rho = - 245\)} in and , and obtain \(\begin{matrix}
{v(t)} & = & {294e^{- t/25} - 245,} \\
{y(t)} & = & {7350 - 245t - 7350e^{- t/25}.}
\end{matrix}\) To find the time required for the bolt to reach its maximum height (when
{\(v = 0\)}), we solve the equation \(v(t) = 294e^{- t/25} - 245 = 0\) for {\(t_{m} = 25\ln(294/245) \approx 4.558\)} (s). Its maximum height
is then {\(y_{\text{max}} = v\left( t_{m} \right) \approx 108.280\)}
meters (as opposed to 122.5 meters without air resistance). To find when
the bolt strikes the ground, we must solve the equation \(y(t) = 7350 - 245t - 7350e^{- t/25} = 0.\) {}Using Newton's method, we can begin with the initial guess
{\(t_{0} = 10\)} and carry out the iteration
{\(t_{n + 1} = t_{n} - y\left( t_{n} \right)/y\text{'}\left( t_{n} \right)\)}
to generate successive approximations to the root. Or we can simply use
the \texttt{Solve} command on a calculator or computer. We find that the
bolt is in the air for {\(t_{\text{f}} \approx 9.411\)} seconds (as
opposed to 10 seconds without air resistance). It hits the ground with a
reduced speed of
{\(\left| v\left( t_{\text{f}} \right) \middle| \approx 43.227\text{m/s} \right.\)}
(as opposed to its initial velocity of 49 m/s). Thus the effect of air resistance is to decrease the bolt's maximum
height, the total time spent aloft, and its final impact speed. Note
also that the bolt now spends more time in descent
({\(t_{\text{f}} - t_{m} \approx 4.853\)} s) than in ascent
({\(t_{m} \approx 4.558\)} s).

\vspace{1em}
\noindent\textbf{2.3.7.}
\par
\textbf{} (a) 100 ft/sec\;;\;\;  (b) about 23 sec and 1403 ft to reach 90
ft/sec

\vspace{1em}
\noindent\textbf{2.3.9.}
\par
\textbf{} 50 ft/s

\vspace{1em}

\end{document}