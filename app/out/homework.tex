
\documentclass{article}
\usepackage[legalpaper, margin=0.5in]{geometry}

\usepackage{textcomp}
\usepackage{fontspec}
\usepackage{unicode-math}
\everymath{\displaystyle\scriptstyle}
\usepackage{cprotect}
\setmainfont{Latin Modern Roman}
\setmathfont{Latin Modern Math}

\usepackage{hyperref}

\usepackage{amsmath}

\relpenalty=10000
\binoppenalty=10000

\setlength{\parindent}{0pt}

\begin{document}

\newlength{\partheight}
\setlength{\partheight}{0.2375\textheight}

\begin{minipage}[t]{\textwidth}
\centering
{\Large \textbf{Homework 2.3-2.3}}
\end{minipage}

\begin{minipage}[t][\partheight]{\textwidth}
\textbf{Ex. 2.3.1.}\quad {No air resistance} Suppose that a crossbow bolt is shot straight upward
from the ground ({$\displaystyle y_{0} = 0$}) with initial velocity {$\displaystyle v_{0} = 49$}
(m/s). Then with {$\displaystyle g = 9.8$} gives $\displaystyle \frac{dv}{dt} = - 9.8,\text{so} v(t) = - (9.8)t + v_{0} = - (9.8)t + 49.$ Hence the bolt's height function \texttt{y}(\texttt{t}) is given by $\displaystyle y(t) = {\int{\left\lbrack - (9.8)t + 49 \right\rbrack dt = - (4.9)t^{2} + 49t + y_{0} = - (4.9)t^{2} + 49t\text{.}}}$ The bolt reaches its maximum height when
{$\displaystyle v = - (9.8)t + 49 = 0\text{,}$} hence when {$\displaystyle t = 5$} (s). Thus its
maximum height is $\displaystyle y_{\max} = y(5) = - (4.9)\left( 5^{2} \right) + (49)(5) = 122.5\left( \text{m} \right)\text{.}$ The bolt returns to the ground when
{$\displaystyle y = - (4.9)t(t - 10) = 0\text{,}$} and thus after 10 seconds aloft.

\end{minipage}
\par
\begin{minipage}[t][\partheight]{\textwidth}
\textbf{Ex. 2.3.2.}\quad {Velocity-proportional resistance} We again consider a bolt shot
straight upward with initial velocity {$\displaystyle v_{0} = 49\text{m/s}$} from a
crossbow at ground level. But now we take air resistance into account,
with {$\displaystyle \rho = 0.04$} in . We ask how the resulting maximum height and
time aloft compare with the values found in .

\end{minipage}
\par
\begin{minipage}[t][\partheight]{\textwidth}
Problems 9 through 12 illustrate resistance proportional to the
velocity.
\\\\
\textbf{Pr. 2.3.7.}\quad Suppose that a car starts from rest, its engine providing an
acceleration of {$\displaystyle 10\text{ft/s}^{2}\text{,}$} while air resistance
provides {$\displaystyle 0.1\text{ft/s}^{2}$} of deceleration for each foot per
second of the car's velocity.

\vspace{0.5em}\hspace*{0.5em}\textbf{(a)} Find the car's maximum
possible (limiting) velocity.

\vspace{0.5em}\hspace*{0.5em}\textbf{(b)} Find how long it takes the
car to attain 90\% of its limiting velocity, and how far it travels
while doing so.

\end{minipage}
\par
\begin{minipage}[t][\partheight]{\textwidth}
Problems 17 and 18 apply -- to the motion of a crossbow bolt.
\\\\
\textbf{Pr. 2.3.9.}\quad A motorboat weighs 32,000 lb and its motor provides a thrust of 5000 lb.
Assume that the water resistance is 100 pounds for each foot per second
of the speed \texttt{v} of the boat. Then $\displaystyle 1000\frac{dv}{dt} = 5000 - 100v.$ If the boat starts from rest, what is the maximum velocity that it can
attain?

\end{minipage}
\par
\newpage
\begin{center}
{\Large \textbf{Answer Key}}
\end{center}
\vspace{1em}
\noindent\textbf{Ex. 2.3.2.}
We substitute {$\displaystyle y_{0} = 0,v_{0} = 49\text{,}$} and
{$\displaystyle v_{\tau} = - g/\rho = - 245$} in and , and obtain $\displaystyle \begin{matrix}
{v(t)} & = & {294e^{- t/25} - 245,} \\
{y(t)} & = & {7350 - 245t - 7350e^{- t/25}.}
\end{matrix}$ To find the time required for the bolt to reach its maximum height (when
{$\displaystyle v = 0$}), we solve the equation $\displaystyle v(t) = 294e^{- t/25} - 245 = 0$ for {$\displaystyle t_{m} = 25\ln(294/245) \approx 4.558$} (s). Its maximum height
is then {$\displaystyle y_{\text{max}} = v\left( t_{m} \right) \approx 108.280$}
meters (as opposed to 122.5 meters without air resistance). To find when
the bolt strikes the ground, we must solve the equation $\displaystyle y(t) = 7350 - 245t - 7350e^{- t/25} = 0.$ {}Using Newton's method, we can begin with the initial guess
{$\displaystyle t_{0} = 10$} and carry out the iteration
{$\displaystyle t_{n + 1} = t_{n} - y\left( t_{n} \right)/y\text{'}\left( t_{n} \right)$}
to generate successive approximations to the root. Or we can simply use
the \texttt{Solve} command on a calculator or computer. We find that the
bolt is in the air for {$\displaystyle t_{\text{f}} \approx 9.411$} seconds (as
opposed to 10 seconds without air resistance). It hits the ground with a
reduced speed of
{$\displaystyle \left| v\left( t_{\text{f}} \right) \middle| \approx 43.227\text{m/s} \right.$}
(as opposed to its initial velocity of 49 m/s). Thus the effect of air resistance is to decrease the bolt's maximum
height, the total time spent aloft, and its final impact speed. Note
also that the bolt now spends more time in descent
({$\displaystyle t_{\text{f}} - t_{m} \approx 4.853$} s) than in ascent
({$\displaystyle t_{m} \approx 4.558$} s).

\vspace{1em}
\noindent\textbf{Pr. 2.3.7.}
\textbf{} (a) 100 ft/sec\;;\;\;  (b) about 23 sec and 1403 ft to reach 90
ft/sec

\vspace{1em}
\noindent\textbf{Pr. 2.3.9.}
\textbf{} 50 ft/s

\vspace{1em}

\end{document}