
\documentclass{article}
\usepackage[legalpaper, margin=0.5in]{geometry}

\usepackage{textcomp}
\usepackage{fontspec}
\usepackage{unicode-math}
\usepackage{cprotect}
\setmainfont{Latin Modern Roman}
\setmathfont{Latin Modern Math}

\usepackage{hyperref}

\usepackage{amsmath}

\setlength{\parindent}{0pt}

\begin{document}

\newlength{\partheight}
\setlength{\partheight}{0.2375\textheight}

\begin{minipage}[t][0.05\textheight]{\textwidth}
\centering
{\Large \textbf{Homework Template}}
\end{minipage}


\begin{minipage}[t][\partheight]{\textwidth}
\textbf{1.4.1}\;\leavevmode\vadjust pre{\hypertarget{P700101346200000000000000000F0D1}{}}%
Problems 54 through 64 illustrate the application of Torricelli's law.

\leavevmode\vadjust pre{\hypertarget{P700101346200000000000000000F0D4}{}}%
A tank is shaped like a vertical cylinder; it initially contains water
to a depth of 9 ft, and a bottom plug is removed at time {\(t = 0\)}
(hours). After 1 h the depth of the water has dropped to 4 ft. How long
does it take for all the water to drain from the tank?

\end{minipage}
\par
\begin{minipage}[t][\partheight]{\textwidth}
\textbf{1.4.7}\;\leavevmode\vadjust pre{\hypertarget{P700101346200000000000000000F0D1}{}}%
Problems 54 through 64 illustrate the application of Torricelli's law.

\leavevmode\vadjust pre{\hypertarget{P700101346200000000000000000F0E0}{}}%
A cylindrical tank with length 5 ft and radius 3 ft is situated with its
axis horizontal. If a circular bottom hole with a radius of 1 in. is
opened and the tank is initially half full of water, how long will it
take for the liquid to drain completely?

\end{minipage}
\par
\begin{minipage}[t][\partheight]{\textwidth}
\textbf{1.4.22}\;\leavevmode\vadjust pre{\hypertarget{P700101346200000000000000000F08D}{}}%
Problems 29 through 32 explore the connections among general and
singular solutions, existence, and uniqueness.

\leavevmode\vadjust pre{\hypertarget{P700101346200000000000000000F0BA}{}}%
{Pollution increase} The amount \texttt{A}(\texttt{t}) of atmospheric
pollutants in a certain mountain valley grows naturally and is tripling
every 7.5 years.

\begin{enumerate}
\item
  \leavevmode\vadjust pre{\hypertarget{P700101346200000000000000000F0BD}{}}%
  If the initial amount is 10 pu (pollutant units), write a formula for
  \texttt{A}(\texttt{t}) giving the amount (in pu) present after
  \texttt{t} years.
\item
  \leavevmode\vadjust pre{\hypertarget{P700101346200000000000000000F0BF}{}}%
  What will be the amount (in pu) of pollutants present in the valley
  atmosphere after 5 years?
\item
  \leavevmode\vadjust pre{\hypertarget{P700101346200000000000000000F0C1}{}}%
  If it will be dangerous to stay in the valley when the amount of
  pollutants reaches 100 pu, how long will this take?
\end{enumerate}

\end{minipage}
\par
\begin{minipage}[t][\partheight]{\textwidth}
\textbf{1.4.24}\;\leavevmode\vadjust pre{\hypertarget{P700101346200000000000000000F08D}{}}%
Problems 29 through 32 explore the connections among general and
singular solutions, existence, and uniqueness.

\leavevmode\vadjust pre{\hypertarget{P700101346200000000000000000F0CC}{}}%
{Growth of languages} There are now about 3300 different human
``language families'' in the whole world. Assume that all these are
derived from a single original language and that a language family
develops into 1.5 language families every 6 thousand years. About how
long ago was the single original human language spoken?

\end{minipage}
\par
\begin{minipage}[t][\partheight]{\textwidth}
\textbf{1.5.3}\;\hypertarget{P700101346200000000000000000F1B7}{}
\hypertarget{P700101346200000000000000000F1B8}{%
\section{\texorpdfstring{{Solution}}{Solution}}}

Division by {\(x^{2}\)} gives the linear first-order equation

\hypertarget{P700101346200000000000000000F1BA}{}
\[\frac{dy}{dx}\text{~+~}\frac{1}{x}\text{~y~=~}\frac{\sin\ x}{x^{2}}\]

with {\(P(x) = 1/x\)} and {\(Q(x) = (\sin\ x)/x^{2}\)}. With
{\(x_{0} = 1\)} the integrating factor in
\protect\hyperlink{P7001013462000000000000000001CE1}{{(12)}} is

\hypertarget{P700101346200000000000000000F1BC}{}
\[\rho(x) = \exp\left( {\int_{1}^{x}{\frac{1}{t}dt}} \right) = \exp(\ln\ x) = x,\]

so the desired particular solution is given by

\hypertarget{P7001013462000000000000000001CEE}{}
{(14)}{\[y(x) = \frac{1}{x}\left\lbrack {y_{0} + {\int_{1}^{x}\frac{\sin\ t}{t}}\ dt} \right\rbrack.\]}

In accord with
\protect Theorem 1,
this solution is defined on the whole positive \texttt{x}-axis.

\hypertarget{P7001013462000000000000000001CE3}{}
Solve the initial value problem

\hypertarget{P7001013462000000000000000001CE6}{}
{(13)}{\(x^{2}\frac{dy}{dx} + xy = \sin\ x,\mspace{25mu} y(1) = y_{0}.\)}

\end{minipage}
\par
\begin{minipage}[t][\partheight]{\textwidth}
\textbf{1.5.5}\;\leavevmode\vadjust pre{\hypertarget{P700101346200000000000000000F253}{}}%
Problems 33 through 37 illustrate the application of linear first-order
differential equations to mixture problems.

\leavevmode\vadjust pre{\hypertarget{P700101346200000000000000000F25E}{}}%
A 400-gal tank initially contains 100~gal of brine containing 50~lb of
salt. Brine containing 1~lb of salt per gallon enters the tank at the
rate of 5~gal/s, and the well-mixed brine in the tank flows out at the
rate of 3~gal/s. How much salt will the tank contain when it is full of
brine?

\end{minipage}
\par
\begin{minipage}[t][\partheight]{\textwidth}
\textbf{1.5.7}\;\leavevmode\vadjust pre{\hypertarget{P700101346200000000000000000F253}{}}%
Problems 33 through 37 illustrate the application of linear first-order
differential equations to mixture problems.

\leavevmode\vadjust pre{\hypertarget{P700101346200000000000000000F269}{}}%
{Two tanks} Suppose that in the cascade shown in
\protect Fig. 1.5.5,
tank 1 initially contains 100~gal of pure ethanol and tank 2 initially
contains 100~gal of pure water. Pure water flows into tank 1 at
10~gal/min, and the other two flow rates are also
10~gal/min.\textbf{(a)} Find the amounts \texttt{x}(\texttt{t}) and
\texttt{y}(\texttt{t}) of ethanol in the two tanks at time
{\(t \geqq 0.\)} \textbf{(b)} Find the maximum amount of ethanol ever in
tank 2.

\end{minipage}
\par
\begin{minipage}[t][\partheight]{\textwidth}
\textbf{1.5.16}\;\leavevmode\vadjust pre{\hypertarget{P700101346200000000000000000F202}{}}%
Find general solutions of the differential equations in Problems 1
through 25. If an initial condition is given, find the corresponding
particular solution. Throughout, primes denote derivatives with respect
to \texttt{x}.

\leavevmode\vadjust pre{\hypertarget{P700101346200000000000000000F223}{}}%
{\(y\text{'} = (1 - y)\ \cos\ x\)}, {\(y(\pi) = 2\)}

\end{minipage}
\par
\begin{minipage}[t][\partheight]{\textwidth}
\textbf{1.5.23}\;\leavevmode\vadjust pre{\hypertarget{P700101346200000000000000000F202}{}}%
Find general solutions of the differential equations in Problems 1
through 25. If an initial condition is given, find the corresponding
particular solution. Throughout, primes denote derivatives with respect
to \texttt{x}.

\leavevmode\vadjust pre{\hypertarget{P700101346200000000000000000F231}{}}%
{\(xy\text{'} + (2x - 3)y = 4x^{4}\)}

\end{minipage}
\par
\begin{minipage}[t][\partheight]{\textwidth}
\textbf{1.5.24}\;\leavevmode\vadjust pre{\hypertarget{P700101346200000000000000000F202}{}}%
Find general solutions of the differential equations in Problems 1
through 25. If an initial condition is given, find the corresponding
particular solution. Throughout, primes denote derivatives with respect
to \texttt{x}.

\leavevmode\vadjust pre{\hypertarget{P700101346200000000000000000F233}{}}%
{\((x^{2} + 4)y\text{'} + 3xy = x\)}, {\(y(0) = 1\)}

\end{minipage}
\par
\begin{minipage}[t][\partheight]{\textwidth}
\textbf{1.5.1}\;\hypertarget{P700101346200000000000000000F177}{}
\hypertarget{P700101346200000000000000000F178}{%
\section{\texorpdfstring{{Solution}}{Solution}}}

Here we have {\(P(x) \equiv - 1\)} and
{\(Q(x) = \frac{11}{8}e^{- x/3}\)}, so the integrating factor is

\hypertarget{P700101346200000000000000000F17A}{}
\[\rho(x) = e^{\int( - 1)dx} = e^{- x}.\]

Multiplication of both sides of the given equation by {\(e^{- x}\)}
yields

\hypertarget{P7001013462000000000000000001C9D}{}
{(7)}{\[e^{- x}\frac{dy}{dx} - e^{- x}y = \frac{11}{8}e^{- 4x/3},\]}

which we recognize as

\hypertarget{P700101346200000000000000000F17D}{}
\[\frac{d}{dx}(e^{- x}y) = \frac{11}{8}e^{- 4x/3}.\]

Hence integration with respect to \texttt{x} gives

\hypertarget{P700101346200000000000000000F17F}{}
\[e^{- x}y = {\int\frac{11}{8}}e^{- 4x/3}dx = - \frac{33}{32}e^{- 4x/3} + C,\]

and multiplication by {\(e^{x}\)} gives the general solution

\hypertarget{P7001013462000000000000000001CA3}{}
{(8)}{\[y(x) = Ce^{x} - \frac{33}{32}e^{- x/3}.\]}

Substitution of {\(x = 0\)} and {\(y = - 1\)} now gives
{\(C = \frac{1}{32},\)} so the desired particular solution is

\hypertarget{P700101346200000000000000000F182}{}
\[y(x) = \frac{1}{32}e^{x} - \frac{33}{32}e^{- x/3} = \frac{1}{32}(e^{x} - 33e^{- x/3}).\]

\hypertarget{P7001013462000000000000000001C94}{}
Solve the initial value problem

\hypertarget{P700101346200000000000000000F176}{}
\(\frac{dy}{dx} - y = \frac{11}{8}e^{- x/3},\mspace{25mu} y(0) = \  - 1.\)

\end{minipage}
\par
\begin{minipage}[t][\partheight]{\textwidth}
\textbf{1.5.2}\;\hypertarget{P700101346200000000000000000F18D}{}
\hypertarget{P700101346200000000000000000F18E}{%
\section{\texorpdfstring{{Solution}}{Solution}}}

After division of both sides of the equation by {\(x^{2} + 1\)}, we
recognize the result

\hypertarget{P700101346200000000000000000F190}{}
\[\frac{dy}{dx} + \frac{3x}{x^{2} + 1}\ y\  = \frac{6x}{x^{2} + 1}\]

as a first-order linear equation with {\(P(x) = 3x/(x^{2} + 1)\)} and
{\(Q(x) = 6x/(x^{2} + 1)\)}. Multiplication by

\hypertarget{P700101346200000000000000000F192}{}
\[\rho(x) = \exp\ \left( {\int{\frac{3x}{x2 + 1}dx}} \right) = \exp\left( {\frac{3}{2}\ln(x^{2} + 1)} \right) = {(x^{2} + 1)}^{3/2}\]

yields

\hypertarget{P700101346200000000000000000F194}{}
\[{(x^{2} + 1)}^{3/2}\frac{dy}{dx} + 3x{(x^{2} + 1)}^{1/2}y = 6x{(x^{2} + 1)}^{1/2},\]

and thus

\hypertarget{P700101346200000000000000000F196}{}
\[D_{x}\left\lbrack {(x^{2} + 1)}^{3/2}y \right\rbrack = 6x{(x^{2} + 1)}^{1/2}.\]

Integration then yields

\hypertarget{P700101346200000000000000000F198}{}
\[{(x^{2} + 1)}^{3/2}y = {\int{6x{(x^{2} + 1)}^{1/2}}}dx = 2{(x^{2} + 1)}^{3/2} + C.\]

Multiplication of both sides by {\({(x^{2} + 1)}^{- 3/2}\)} gives the
general solution

\hypertarget{P7001013462000000000000000001CC1}{}
{(10)}{\[y(x) = 2 + C{(x^{2} + 1)}^{- 3/2}.\]}

\hypertarget{P7001013462000000000000000001CAF}{}
Find a general solution of

\hypertarget{P7001013462000000000000000001CB3}{}
{(9)}{\((x^{2} + 1)\frac{dy}{dx} + 3xy = 6x.\)}

\end{minipage}
\par

\end{document}